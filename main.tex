% $Header$

\documentclass{beamer}

% This file is a solution template for:

% - Talk at a conference/colloquium.
% - Talk length is about 20min.
% - Style is ornate.



% Copyright 2004 by Till Tantau <tantau@users.sourceforge.net>.
%
% In principle, this file can be redistributed and/or modified under
% the terms of the GNU Public License, version 2.
%
% However, this file is supposed to be a template to be modified
% for your own needs. For this reason, if you use this file as a
% template and not specifically distribute it as part of a another
% package/program, I grant the extra permission to freely copy and
% modify this file as you see fit and even to delete this copyright
% notice. 


\mode<presentation>
{
  \usetheme{Warsaw}
  % or ...

  \setbeamercovered{transparent}
  % or whatever (possibly just delete it)
}


\usepackage[english]{babel}
% or whatever

\usepackage[utf8]{inputenc}
% or whatever

\usepackage{times}
\usepackage[T1]{fontenc}
% Or whatever. Note that the encoding and the font should match. If T1
% does not look nice, try deleting the line with the fontenc.

%\newtheorem{definition}{Definition}

\title% (optional, use only with long paper titles)
{Consensus and replication}

\subtitle
{From impossibility to production}

\author[Alexander Kharitonov]{Alexander Kharitonov \\ \texttt{a@cmpxchg.me}}
\date{Yandex science seminar, 22 Mar 2012}

%\author[Author, Another] % (optional, use only with lots of authors)
%{F.~Author\inst{1} \and S.~Another\inst{2}}
% - Give the names in the same order as the appear in the paper.
% - Use the \inst{?} command only if the authors have different
%   affiliation.
%\institute[Yandex]
%{
%    Yandex LLC
%}

%\institute[Universities of Somewhere and Elsewhere] % (optional, but mostly needed)
%{
%  \inst{1}%
%  Department of Computer Science\\
%  University of Somewhere
%  \and
%  \inst{2}%
%  Department of Theoretical Philosophy\\
%  University of Elsewhere}
% - Use the \inst command only if there are several affiliations.
% - Keep it simple, no one is interested in your street address.

%\date[CFP 2003] % (optional, should be abbreviation of conference name)
%{Conference on Fabulous Presentations, 2003}
% - Either use conference name or its abbreviation.
% - Not really informative to the audience, more for people (including
%   yourself) who are reading the slides online

\subject{Distributed systems}
% This is only inserted into the PDF information catalog. Can be left
% out. 



% If you have a file called "university-logo-filename.xxx", where xxx
% is a graphic format that can be processed by latex or pdflatex,
% resp., then you can add a logo as follows:

% \pgfdeclareimage[height=0.5cm]{university-logo}{university-logo-filename}
% \logo{\pgfuseimage{university-logo}}



% Delete this, if you do not want the table of contents to pop up at
% the beginning of each subsection:
\AtBeginSubsection[]
{
  \begin{frame}<beamer>{Outline}
    \tableofcontents[currentsection,currentsubsection]
  \end{frame}
}


% If you wish to uncover everything in a step-wise fashion, uncomment
% the following command: 

%\beamerdefaultoverlayspecification{<+->}


\begin{document}

\begin{frame}
  \titlepage
\end{frame}

\begin{frame}{Outline}
  \tableofcontents
  % You might wish to add the option [pausesections]
\end{frame}


% Structuring a talk is a difficult task and the following structure
% may not be suitable. Here are some rules that apply for this
% solution: 

% - Exactly two or three sections (other than the summary).
% - At *most* three subsections per section.
% - Talk about 30s to 2min per frame. So there should be between about
%   15 and 30 frames, all told.

% - A conference audience is likely to know very little of what you
%   are going to talk about. So *simplify*!
% - In a 20min talk, getting the main ideas across is hard
%   enough. Leave out details, even if it means being less precise than
%   you think necessary.
% - If you omit details that are vital to the proof/implementation,
%   just say so once. Everybody will be happy with that.

\section{Motivation}
\subsection{High availability and state machine replication}

\begin{frame}{Introduction}
  \begin{itemize}
  \item We want to build systems that work despite some number of failing components.
  \item Particularly want no single points of failure.
  \item Hence such systems are necessarily distributed.
  \item Unfortunately, distributed systems are very hard to reason about.
  \item Separate the resilience and application logic: treat the system as a state machine.
  \end{itemize}
\end{frame}

\begin{frame}{State machine}
  \begin{definition}
    A state machine is a tuple $S = (Q, \varphi, I, O)$, where $Q$ is the set of states, $\varphi\colon Q \times I \to Q \times O$ is the state transition function, $I$ is the input command set and $O$ is the output set.
  \end{definition}
  Some (trivial) notes:
  \begin{itemize}
  \item $Q$ is not necessarily finite, so e.g. a Turing machine is a valid state machine by our definition.
  \item $\varphi$ is deterministic.
  \end{itemize}
\end{frame}

\begin{frame}{State machine example: a read-write register}
  \begin{definition}
    A read-write register is essentially a memory cell that can hold a single natural number (or be empty).
    \begin{itemize}
      \item $Q = \mathbb{N} \cup \{ \bot \}$
      \item $I = \textsc{read} \cup \{ \textsc{write}(i) \mid i \in \mathbb{N} \}$
      \item $O = \textsc{ok} \cup Q$
      \item $\varphi(q, \textsc{read}) = (q, q)$
      \item $\varphi(q, \textsc{write}(i)) = (i, \textsc{ok})$
    \end{itemize}
  \end{definition}
\end{frame}

%\begin{frame}{Sequential specifications and linearizability}
%  \begin{itemize}
%    \item A state machine is a complete sequential specification of the system that it models.

\begin{frame}{State machine replication}
  \begin{itemize}
    \item In real life computers crash, power goes out etc. Systems fail.
    \item To remain available despite faults the system must be replicated.
  \end{itemize}
\end{frame}

\begin{frame}{State machine replication: the setting}
  \begin{itemize}
    \item There is a set of clients $C$ and a set of replicas $R$ which communicate over the network.
    \item Some protocol provides the clients with an abstraction of a single state machine they all interact with (backed by $R$).
    \item Can have various consistency guarantees depending on the system model and on kinds of tolerated failures, e.g. linearizability, single-client FIFO, eventual consistency etc.
    \item Tolerates some kinds of replica failures (again, depends on the system model and synchrony assumptions).
  \end{itemize}
\end{frame}

\subsection{Replication, atomic broadcast and consensus}

\begin{frame}{On failure model and synchrony assumptions}
  We will assume the following:
  \begin{itemize}
    \item Failures are crash-stop: no crashed replica ever recovers. This is for simplicity, practically everything can be adapted to the crash-recovery setting.
    \item The system is initially asynchronous. That means no bounds on message losses and delays, no bounds on the relative speed of the processes, no shared clocks whatsoever.
    \item There is an (unknown to the processes) Global Stabilization Time, after which no processes crash and message delays and clock drift become bounded (but bounds remain unknown).
  \end{itemize}
\end{frame}

\begin{frame}{On failure model and synchrony assumptions (cont'd)}
  \begin{itemize}
    \item Actually, things need not to remain stable forever after GST. The only requirement for liveness is for the stability period to be long enough for a step of replication to complete.
    \item Under these assumptions we will construct systems that tolerate up to $\lfloor\frac{n-1}{2}\rfloor$ failures in a system of $n$ processes.
  \end{itemize}
  \begin{definition}
    With respect to some execution of some distributed algorithm, a \alert{correct} process is a process that did not crash in that execution.
  \end{definition}
\end{frame}

\begin{frame}{Atomic broadcast}
  \begin{definition}
    Atomic broadcast is a group communication primitive providing two operations:
    \begin{itemize}
      \item $\textsc{broadcast}(m)$ -- send a message $m$ to everyone in the group
      \item $\textsc{deliver}(m)$ -- deliver a message.
    \end{itemize}
    with the following guarantees:
  \end{definition}
\end{frame}

\begin{frame}{Atomic broadcast guarantees}
  \begin{enumerate}
    \item If a process \textsc{deliver}s $m$, then all \alert{correct} processes eventually (after GST) deliver $m$.
    \item A process can only \textsc{deliver} a message that has been \textsc{broadcast} by some process.
    \item No two processes \textsc{deliver} any two messages in different orders.
    \item If a correct process \textsc{broadcast}s $m$, then all \alert{correct} processes eventually \textsc{deliver} $m$.
  \end{enumerate}
  Guarantees $2$ and $3$ are \alert{uniform safety} properties that hold for \alert{all} processes, including the faulty ones.
\end{frame}

\begin{frame}{State machine replication via atomic broadcast}
  \begin{itemize}
    \item With atomic broadcast state machine replication becomes easy, as it delivers a uniform sequence of message to all replicas.
    \item A na\"ive but working approach to SMR would be for each client to maintain a copy of the state machine, \textsc{broadcast} its commands and apply commands in the order they are \textsc{deliver}ed.
  \end{itemize}
\end{frame}

\begin{frame}{Consensus}
  \begin{definition}
    Consensus is a distributed agreement primitive defined by two operations
    \begin{itemize}
      \item $\textsc{propose(v)}$ -- propose a value $v$
      \item $\textsc{decide(v)}$ -- decide that $v$ has been chosen
    \end{itemize}
    satisfying:
    \begin{enumerate}
      \item If a process \textsc{decide}s $v$ then some process \textsc{propose}d $v$
      \item \textsc{decide} \alert{never} decides two different values (in a single consensus execution, there are no two \textsc{decide} calls returning different values).
      \item If any correct process \textsc{propose}s a value then eventually all correct processes \textsc{decide} some value.
    \end{enumerate}
  \end{definition}
\end{frame}

\begin{frame}{Consensus is equivalent to atomic broadcast}{Consensus via atomic broadcast}
  \begin{enumerate}
    \item To \textsc{propose} a value $v$, \textsc{broadcast}($v$).
    \item \textsc{decide} on a first value \textsc{deliver}ed.
  \end{enumerate}
\end{frame}

\begin{frame}{Consensus is equivalent to atomic broadcast}{Atomic broadcast via consensus}
  \begin{itemize}
    \item Run a series of consensus instances indexed by $\mathbb{N}$.
    \item Intuitively, consensus in instance $i$ on a value $m$ means that $m$ is the $i$th value in the broadcast sequence.
    \item To \textsc{broadcast} a value $v$, pick the lowest instance on which consensus is unknown and \textsc{propose} $v$. Repeat until success.
    \item \textsc{deliver} values in the order of their instance numbers.
  \end{itemize}
\end{frame}

\section{Impossibility results}
\subsection{Asynchronous model}
%\begin{frame}{Atoi


\end{document}


